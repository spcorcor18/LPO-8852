

% *****************************************************************************
% Regression II - Lecture 6 In-Class Exercise (Synthetic Control)
% 
% Last updated: October 27, 2025
% *****************************************************************************

\documentclass[12pt]{article}
\usepackage{amsmath}
\usepackage{amsfonts}
\usepackage{amssymb}
\usepackage[bookmarks=false]{hyperref}
\usepackage{setspace}%
\usepackage{booktabs}           % needed for professional tables
\usepackage{graphicx}

\newlength{\toppush}
\setlength{\toppush}{2\headheight}
\addtolength{\toppush}{\headsep}

\newcommand{\htitle}[3]{\noindent\vspace*{-\toppush}\newline\parbox{6.5in}
   {Regression II\hfill\newline
    Vanderbilt University    \hfill #2 \newline
    Prof. Sean P. Corcoran \hfill #3 \newline
    \mbox{}\hrulefill\mbox{}}\vspace*{0.4ex}\mbox{}\newline
    \begin{center}{\bf #1}\end{center}}

\newcommand{\handout}[3]{\thispagestyle{empty}
\pagestyle{myheadings}\htitle{#1}{#2}{#3}}
   % \thispagestyle{empty} indicates no page numbers for this (the first) page

\newcommand{\simnot}{\mathord{\sim}}  % needed for correct spacing of ~

\setlength{\oddsidemargin}{0pt}
\setlength{\evensidemargin}{0pt}
\setlength{\textwidth}{6.5in}
\setlength{\topmargin}{0in}
\setlength{\textheight}{8.5in}

\graphicspath{{Graphics/}}

\begin{document}


% *****************************************************************************

\handout{Lecture 6 In-Class Exercise}{}{Last updated: October 27, 2025}
\setlength{\parindent}{0pt}

\hrulefill
\bigskip

% *****************************************************************************
% Mixtape synthetic control example
% *****************************************************************************
\bigskip

This exercise will replicate and expand upon the results presented in the \textit{Mixtape} chapter on Synthetic Control. In response to a civil action lawsuit related to prison overcrowding, Texas significantly expanded its state prison capacity in the 1980s. The growth in prison capacity increased rapidly beginning in 1993 when the state approved \$1 billion in new prison construction. In the \textit{Mixtape}, Cunningham examines the effect of new prison construction on the incarceration rates of Black men. There is evidence that Black men are systematically more likely to be incarcerated (even for the same infractions), and that parole decisions are racially biased. One might predict, then, that the expanded capacity of prisons in Texas had a disproportionate effect on Black men. Cunningham uses synthetic control to address this question. (In \href{www.scunning.com/files/mass_incarceration_and_drug_abuse.pdf}{this working paper}, he looks at the effect of higher incarceration rates on drug markets). The dataset he uses can be read into Stata as follows:
\bigskip

\texttt{use https://github.com/scunning1975/mixtape/raw/master/texas.dta, clear}
\bigskip

See the accompanying do-file for Stata syntax.

\begin{enumerate}
\setlength\itemsep{1em}

\item Part 1 of the accompanying do-file uses the \texttt{synth2} command to construct a synthetic control and estimate the effect of prison construction in Texas on the incarceration rates of Black men (\textit{bmprate,} the number incarcerated per 100,000 population). The above dataset has already been ``xtset'' with \textit{statefip} as the cross-sectional unit and \textit{year} as the time period. The state ID for Texas is 48 and the first treatment year is assumed to be 1993. Cunningham selected 14 predictor variables, a combination of pre-treatment outcomes and covariates:

\begin{itemize}
\item[] Incarceration rate in 1988, 1990, 1991, and 1992
\item[] Alcohol (1990), and AIDS per capita (1990 and 1991) -- not sure what these are, or why they are used (I think this analysis is part of a larger study)
\item[] Income, unemployment rate, poverty rate -- \textit{averages over the full pre-treatment period}
\item[] Percent Black (1990, 1991, and 1992)
\item[] Percent aged 15-19 (1990)
\end{itemize}

From \texttt{synth2} you get the cross-sectional unit weights (i.e., which states are used as the synthetic control and the weights they are given), coefficient ($v$) weights, a pre-treatment balance table for the predictors, post-treatment actual and synthetic outcomes, and a series of graphs (see below). See \href{https://journals.sagepub.com/doi/10.1177/1536867X231195278}{Yan \& Chen (2023)} for details on \texttt{synth2}.

\begin{figure}[h]
\begin{center}
\includegraphics[width=0.65\textwidth]{synth2_set1.png}
\end{center}
\end{figure}

\item Part 2 of the do file uses \texttt{synth2} to obtain placebo tests for inference and a ``leave-one-out'' (LOO) robustness test. There are two types of placebo tests: ``in-space'' and ``in-time.'' The in-space test estimates the synthetic control model for every \textit{other} cross-sectional unit (here, state) and compares the results to focal model (here, Texas). The in-time test estimates the synthetic control model using a different time period as the treatment period (here, 1989). For the in-space test, \texttt{synth2} will give you the pre- and post-treatment RMSPEs (and the ratio of post-to-pre) for each unit (state). It also produces ``p-values'' that tell you how unlikely your result would be if it occurred by chance. You also get some useful graphs, like the one below showing the SCM results for each state. Note for the in-space placebo test it is possible to omit placebo units where the RMSPE fit is especially poor. (This is also shown in the do-file). \\

For the in-time test, \texttt{synth2} will give you the actual and synthetic control outcomes (and ``treatment effect'') in each period, assuming the ``fake'' treatment time. \\

The LOO robustness test repeatedly re-estimates the synthetic control model by iteratively excluding one of the units used in the original synthetic control. The resulting graph shows these results overlaid on one another.


\pagebreak
\item Parts A-G of the do file are only for your reference. They show how to use the older \texttt{synth} and \texttt{synth\_runner} commands to accomplish what \texttt{synth2} does. (\texttt{synth2} only works with Stata 16+). It also includes a manual creation of the graph between Texas and its synthetic control, using data saved from \texttt{synth}. This code is not necessary with \texttt{synth2}, but shows how you can take the results and produce your own graph. Likewise, it shows how you can manually conduct the placebo inference tests (unnecessary with \texttt{synth2}).\\

\textbf{Graphs for Part 2:}


\begin{figure}[h!]
\begin{center}
\includegraphics[width=0.85\textwidth]{synth2_set2a.png}
\end{center}
\end{figure}


\begin{figure}[h!]
\begin{center}
\includegraphics[width=0.85\textwidth]{synth2_set2alt.png}
\end{center}
\end{figure}

\pagebreak%
\begin{figure}[h!]
\begin{center}
\includegraphics[width=0.85\textwidth]{synth2_set3.png}
\end{center}
\end{figure}

\begin{figure}[h!]
\begin{center}
\includegraphics[width=0.85\textwidth]{synth2_set4.png}
\end{center}
\end{figure}



\end{enumerate}

.



\end{document}

